\documentclass{article}

\usepackage[utf8]{inputenc}
\usepackage{unicode-helper}
\usepackage{amsmath}
\usepackage{amssymb}
\usepackage{amsfonts}
\usepackage{microtype}
\usepackage[ngerman]{babel}
\usepackage{amsthm}
\usepackage{mathtools}
\usepackage{braket}
\usepackage{csquotes}
\usepackage[pdftex]{hyperref}
\usepackage{cleveref}

\newcommand{\titlevar}{Geschichten}
\newcommand{\authorvar}{Günthner}
\newcommand{\datevar}{WS24}
\title{\titlevar}
\author{\authorvar}
\date{\datevar}
\hypersetup{
	pdftitle=\titlevar,
	pdfauthor=\authorvar,
	pdfcreationdate=\datevar,
}
\setlength{\parindent}{0pt}

\newcommand{\ms}{\medskip}

\begin{document}
	\maketitle

	Ich weiß, er ist nicht echt, aber er steht dort. Mir gegenüber. Hinter dem Gemüse. Ich weiß, er ist schon lange tot. Ich weiß, er ist nicht echt. Ich weiß, er kann mir nicht mehr wehtun. Also ver suche ich ruhig zu bleiben. Vielleicht sollte ich mit meiner Psychaterin besprechen, ob eine höhere Dosis an Antipsychotikum angemessen wäre...

	\ms

	Ich bin Nathalie, 20 Jahre leide ich schon auf dieser Welt. Wenigstens ist diese schön. Ich gehe aus dem dimm beleuchteten Supermarkt in das gleißende Licht des Tages. Es ist immer wieder ein wunderbarer Anblick, auch wenn er mich gerade nicht aus meinem Loch holen kann. Ich lasse meine müden und grauen Augen über meine Umgebung schweifen. Langsam verändere ich sie. Jetzt ist kein Sonnenschein mehr da, es ist London, 1780 oder so. Alle Leute sind richtig altertümlich angezogen, und begrüßen einander steif, ich muss kichern. Aber zurück zur Gegenwart, ich sollte versuchen, nach Hause zu kommen, also renne ich die Treppe zur U-Bahn hinunter und unten fährt praktischerweise gerade das nächste Raumschiff ein, richtig futuristisch. Jetzt fehlt in der Menschenmenge nur noch der eine oder andere Android. Leider fährt jetzt auch aus der anderen Richtung eine U-Bahn ein, die eher zu London, 1780 passen würde. Naja.

	\ms

	Es ist mal wieder Eng, Abendverkehr halt. Ich habe mich mal wieder in eine Ecke direkt an die Tür gestellt und versuche mich auf meine Arbeit zu konzentrieren, Masterarbeit um genau zu sein, in Mathematik. Aber es ist nun mal von und laut und die Menschen sind laut und ihre Gesichter sind laut. Ich komme kaum darum herum, ihre Gespräche mitzuhören.

	\ms

	\enquote{...könnten Sie das für mich machen....}—\enquote{...bla bla bla....}—\enquote{...witzig, oder?}—diese letzte Konversation ergriff meine Aufmerksamkeit, weil sie mich wütend machte—\enquote{Sie hat sich wirklich Mühe gegeben und sie hat die ganzen Sachen wirklich gut gelernt, trotzdem konnte sie den Beweis nicht durchführen.}—\enquote{Als Frau kann sie sich halt solche Sachen nicht so gut vorstellen, sie sollte halt sich etwas anderes aussuchen als pure Mathematik.}—Wohl zwei Doktoranden, mit einem alten Klischee. Leider zu häufig in unserer Zunft, vor allem bei den älteren, allerdings wohl auch bei den (eher) jüngeren.
\end{document}
